\chapter{Introduction}

Procedural generation enables designers to create content algorithmically instead of manually. This technique is used in entertainment areas, such as, 3D animation and video game design.

One particular case in which procedural generation is used is the creation of random terrains, which is the focus of this thesis.

%\begin{itemize}
%	\item Procedural generation
%    \item Algorithmic approach
%    \item Random Algorithms
%	\item Use Cases such as:
%	\begin{itemize}
%		\item 3D animations
%		\item Video Games
%	\end{itemize}
%\end{itemize}

\section {Problem}

The main advantage of procedural content against manually created content is that, the former, is less time consuming and, as a result, enables designers to create detailed content faster. On the other hand the existent methods for random terrain generation are difficult to control and, sometimes, the parameters are unintuitive, which leads to a trial and error design approach, in which the designer changes the parameters and sees what happens.

\section{Main Requirements}

The main requirements for this application can be divided in technical and functional requirements.

In terms of technical requirements the application must be developed using web based technologies and the random surface generation process must be based in fractal geometry. For the functional requirements, the process must use a CAD surface as a base for the terrain and the details should be generated procedurally.

\section{Proposed Solution}

In this thesis a hybrid approach to content creation is proposed. This approach involves the modelling of a deterministic cad model that provides the basic topology of the terrain and the generation of a random terrain that will provide the details for that topology. This two surfaces are then blended together using a method that is explained in Chapter \ref{chap:methodology}. The main focus of this project is the implementation of the blending process and, due to this fact, the developed application does not deal with the cad modelling phase, but only with the random generation and blending phases.

%\begin{itemize}
%	\item Define basic terrain features with a deterministic cad model
%	\item Generate a random surface that will provide the details
%	\item Blend both surfaces creating a terrain with the basic topology of the cad surface and the high level details of the random surface
%\end{itemize}

\section{Thesis Outline}

This thesis is divided in 6 Chapters. In Chapter \ref{chap:literature_review} some mathematical background is presented and the current available procedural terrain generation methods are reviewed. In Chapter \ref{chap:methodology} the proposed method and the parameters that are available to the user are explained. In Chapter \ref{chap:software_architecture} the architecture of the developed system is specified with the usage of the "4+1" View Model proposed in \cite{Kruchten1995}. In Chapter \ref{chap:results} some outputs of the system are presented and the measurements obtained by the performance benchmarks are discussed. Finally in Chapter \ref{chap:conclusions} the implemented features and some further work are discussed. 