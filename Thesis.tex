%% LaTeX_Thesis_Template.tex
% An unofficial LaTeX template for Cranfield theses.

%%
% This document is an example of the use of the unofficial "cranfieldthesis" 
% LaTeX style file.  I hope it's useful, and good like.

%% Options for the bibliography generator
%% \bib_opts[bib_file]{library.bib}
%% \bib_opts[output_file]{bibliography.tex}

\documentclass[12pt, oneside]{book}

\usepackage{luatex85}

\usepackage[pdftex, pdfpagelabels=true, linktocpage, hidelinks]{hyperref}
% \usepackage{hypcap}

% Use the custom "cranfieldthesis" LaTeX style file. 
\usepackage{cranfieldthesis}
\usepackage{amssymb}
\usepackage{geometry}
\usepackage[titletoc]{appendix}
\usepackage{multirow}
\usepackage{tabularx}
\usepackage{lscape}
\usepackage{pdflscape}
\usepackage{rotating}
\usepackage{titletoc}
\usepackage{tikz}
\usepackage[bottom]{footmisc}
\usepackage{float}
\usetikzlibrary{shapes, arrows, matrix}
\usepackage{adjustbox}
\usepackage{cellspace}
\setlength\cellspacetoplimit{4pt}
\setlength\cellspacebottomlimit{4pt}


\usepackage{scrlayer}

\DeclareNewLayer[
background,
rightmargin,
contents={%
	\parbox[b][\layerheight][c]{\dimexpr\footskip+\footheight\relax}{%
		\hfill\rotatebox{90}{\pagemark}}}
]{lscape.foot}

\DeclareNewLayer[
background,
textarea,
addhoffset=\dimexpr-\headsep-\headheight\relax,
width=\dimexpr\headsep+\headheight\relax,
%contents={\hfill\rotatebox{90}{\headmark}\hspace*{\headsep}}
contents={}
]{lscape.head}

\DeclareNewPageStyleByLayers{lscape}{lscape.foot,lscape.head}

\newcolumntype{b}{X}
\newcolumntype{s}{>{\hsize=.5\hsize}X}
\newcommand{\heading}[1]{\multicolumn{1}{c|}{\textbf{#1}}}

% By default, LaTeX uses a serif font - these are traditionally thought to be
% easier to read.   If you'd prefer sans-serif, please uncomment the 
% following line.
%\renewcommand{\familydefault}{\sfdefault}
\renewcommand{\rmdefault}{phv} % Arial
\renewcommand{\sfdefault}{phv} % Arial

\let\oldappendices\appendices
\let\endoldappendices\endappendices

\renewenvironment{appendices}{
	\oldappendices
		
	\addtocontents{toc}{\protect\setcounter{tocdepth}{1}}
	\renewcommand\thesection{\Alph{section}}%
	
	\titlecontents{section}
	[1.0em]
	{}
	{\appendixname~\thecontentslabel~}
	{}
	{\titlerule*[0.3pc]{.}\contentspage}
	
	\titleformat{\section}{\large\bfseries}{\appendixname~\thesection}{0.5em}{}
	\titleformat{\subsection}{\normalsize\bfseries}{\thesubsection}{0.5em}{}
	\titleformat{\subsubsection}{\normalsize\bfseries}{\thesubsubsection}{0.5em}{}
}{
	\endoldappendices
}

% Switch [1] for 1. in Reference list but maintain [1] in citation


\makeatletter\renewcommand{\@biblabel}[1]{#1.}\makeatother

\newenvironment{notes}{
	\vspace{2em}
	\begin{itemize}
	\footnotesize
	\color{red}
}
{
	\end{itemize}
	\vspace{2em}
}

% Example parameters for a typical taught MSc course
\title{A Web Based Terrain Modeller using Fractals and CAD}
\author{Miguel Marques}
\date{August 2016}
\school{Aerospace, Transport and Manufacturing}
\course{Computational and Software Techniques in Engineering \\ Digital Signal and Image Processing}
\degree{MSc}
\academicyear{2015--2016}
\supervisor{Dr Peter Sherar}
\copyrightyear{2016}


\begin{document}


%% Front matter
%
% This is where we do the title page, etc.
%

\frontmatter

% Standard-Form Title Pages
\maketitle

\pagestyle{plain}

% Abstract and Keywords
\begin{abstract}
	\doublespacing
	
    The process of modelling man made objects is, classically, a manual process where a user deforms a 3D object though the use of several techniques. This process is time-consuming and, with the usage of procedural techniques, can be improved, enabling the creation of more realistic and detailed content. One such case where this techniques have been widely used is in the creation of virtual terrains, not only because of the realistic details, but also due to the possibility of creating huge random infinite terrains that are used in the video game industry.
    Procedural generation has the disadvantage of loss of detailed control over the modelled terrain, that is, using a purely procedural method it is complicated to specify where the major features of the terrain should be positioned. This is the main problem that this thesis addresses.
    
    In this project a hybrid terrain modelling process is proposed. In this process the user creates a simplified version of the desired terrain containing the basic topology, which is then blended with a random generated surface to create a detailed version of the terrain using image processing techniques. Additionally a web application, where this process is implemented, was developed using \textit{Angular.js} and \mbox{\textit{WebGL}}. The architecture for this application is detailed in this thesis. In order to improve the performance of the random generation and the blending process some GPU computations were implemented using \mbox{\textit{WebGL 2.0}}. This computations were then benchmarked and the comparison with their CPU counterparts is shown in this document.
    
    

	\subsubsection*{Keywords}
	Procedural Generation; Terrain Modelling; Web Application; WebGL; Fractal geometry; GPU; Fourier filtering; Noise Synthesis
\end{abstract}

% Acknowledgements
\chapter{Acknowledgements}

\begin{notes}
  \item The author would like to thank
\end{notes}

% Table of Contents
\sstableofcontents

% List of Figures
\sslistoffigures

% List of Tables
\sslistoftables

% List of Equations
\sslistofequations

% The list of abbreviations can't be automatically generated so you need to populate it yourself
\begin{listofabbreviations}
    \abbrev{fBm}{Fractional Brownian Motion}
%    \abbrev{PMHCI}{Picewise Monotone Hermite Cubic Interpolation}
    \abbrev{HPF}{High-pass Filter}
    \abbrev{FFT}{Fast Fourier Transform}
    \abbrev{IFFT}{Inverse Fast Fourier Transform}
\end{listofabbreviations}


%% Main Matter
%
% This is where we include the main thesis content.
%
\mainmatter

\chapter{Introduction}

Procedural generation enables designers to create content algorithmically instead of manually. This technique is used in entertainment areas, such as, 3D animation and video game design.

One particular case in which procedural generation is used is the creation of random terrains, which is the focus of this thesis.

%\begin{itemize}
%	\item Procedural generation
%    \item Algorithmic approach
%    \item Random Algorithms
%	\item Use Cases such as:
%	\begin{itemize}
%		\item 3D animations
%		\item Video Games
%	\end{itemize}
%\end{itemize}

\section {Problem}

The main advantage of procedural content against manually created content is that, the former, is less time consuming and, as a result, enables designers to create detailed content faster. On the other hand the existent methods for random terrain generation are difficult to control and, sometimes, the parameters are unintuitive, which leads to a trial and error design approach, in which the designer changes the parameters and sees what happens.

\section{Main Requirements}

The main requirements for this application can be divided in technical and functional requirements.

In terms of technical requirements the application must be developed using web based technologies and the random surface generation process must be based in fractal geometry. For the functional requirements, the process must use a CAD surface as a base for the terrain and the details should be generated procedurally.

\section{Proposed Solution}

In this thesis a hybrid approach to content creation is proposed. This approach involves the modelling of a deterministic cad model that provides the basic topology of the terrain and the generation of a random terrain that will provide the details for that topology. This two surfaces are then blended together using a method that is explained in Chapter \ref{chap:methodology}. The main focus of this project is the implementation of the blending process and, due to this fact, the developed application does not deal with the cad modelling phase, but only with the random generation and blending phases.

%\begin{itemize}
%	\item Define basic terrain features with a deterministic cad model
%	\item Generate a random surface that will provide the details
%	\item Blend both surfaces creating a terrain with the basic topology of the cad surface and the high level details of the random surface
%\end{itemize}

\section{Thesis Outline}

This thesis is divided in 6 Chapters. In Chapter \ref{chap:literature_review} some mathematical background is presented and the current available procedural terrain generation methods are reviewed. In Chapter \ref{chap:methodology} the proposed method and the parameters that are available to the user are explained. In Chapter \ref{chap:software_architecture} the architecture of the developed system is specified with the usage of the "4+1" View Model proposed in \cite{Kruchten1995}. In Chapter \ref{chap:results} some outputs of the system are presented and the measurements obtained by the performance benchmarks are discussed. Finally in Chapter \ref{chap:conclusions} the implemented features and some further work are discussed. 
\chapter{Literature Review} \label{chap:literature_review}

Procedural generation of terrains has been the focus for many graphics researchers for some time now. Through the years this research has been mostly based on fractional Brownian motion and its similarity to a skyline of mountains, first noticed by Mandelbrot in \cite{Mandelbrot1983}. 

\section {Mathematical Background}

\subsection{Fractal Dimension}

\begin{quotation}
	"A fractal is by definition a set for which the Hausdorff Besicovitch dimension strictly exceeds the topological dimension." \cite[p.15]{Mandelbrot1983}
\end{quotation}

The Hausdorff Besicovitch dimension, also called fractional or fractal dimension, is a measure used to characterize a fractal. This dimension $ D_f $ does not need to be an integer and, for a surface in $ \mathbb{R}^n $, $ D_f \in [n, n+1)$.

\subsection{Fractional Brownian Motion}

Fractional Brownian Motion (fBm) was introduced by Mandelbrot and van Ness in \cite{Mandelbrot1968} and it is an extension of Brownian Motion that uses the Hurst Exponent ($H$) as a parameter to control the correlation between successive values, such that, $ 0 < H < 1 $. More specifically if $H = 0.5$ then fBm is just normal Brownian Motion and, due to that, the increments are independent and not correlated; if $H > 0.5$ the increments have positive correlation and this results in smooth curves while if $H < 0.5$ the increments have negative correlation and this results in erratic curves. \cite{Musgrave1993}

In the field of fractal terrain generation fBm is approximated by $1/f^\beta$ noise, where $\beta$ is the spectral exponent of the noise. Considering $D_f$ as the fractal dimension and $D_E$ as the Euclidean dimension, $1/f^\beta$ noise follows the following rule:
\[
D_f = D_E + 1 - H = D_E + \frac{3-\beta}{2}
\]

\section{Terrain Modelling}

In this section the different methods used for terrain generation are analysed. All the analysed methods try to approximate fBms and they are divided in six categories:
\begin{itemize}
	\item Poisson Faulting
	\item Subdivision Methods
	\item Fourier Filtering
	\item Successive Random Additions
	\item Noise synthesis
	\item Generalized Stochastic Subdivision
\end{itemize}

%\begin{quotation}
%	"the small spectral sums used to create random fractals for computer graphics allow the character of the basis function to show through clearly in the result. Usually, the choice of basis function is implicit in the algorithm: it is a sine wave for Fourier synthesis, a sawtooth wave in polygon subdivision, and a piecewise-cubic Hermite spline in noise-based procedural fBm. You could use a Walsh transform and get square waves as your basis. Wavelets (Ruskai 1992) promise to provide a powerful new set of finite basis functions. And again, sparse convolution (Lewis 1989) or fractal sum of pulses (Lovejoy and Mandelbrot 1985) offer perhaps the greatest flexibility in choice of basis functions." \cite[pg 497]{Ebert2003}
%\end{quotation}

\subsection{Poisson Faulting}

The Poisson Faulting technique, also known as Random cuts algorithm, consists in applying Gaussian random displacements to a plane at Poisson distributed intervals. \cite{Musgrave1989}. This method was initially applied by Mandelbrot in \cite{Mandelbrot1983}. Although this method has the advantage of working for both planes and spheres, in the creation of random planets, it is $O(n^3)$ complex in time.

\subsection{Subdivision Methods}

The methods presented in this section derive from the midpoint displacement algorithm that consists in successively subdividing a line and displacing the division points. These methods are classified in two categories \cite{Mandelbrot1988}:
\begin{itemize}
	\item Wire-frame Midpoint Displacement
	\item Tile Midpoint Displacement
\end{itemize}

\subsubsection{Wire-frame Midpoint Displacement} \label{sect:wireframeMD}

In this type of methods the surface is thought of as a wire-frame mesh and the displacements are applied to the midpoints of the edges. This is the case of the Carpenter's Method \cite{Fournier1982} in which the wire-frame forms triangles which are recursively subdivided until there is no triangle with a side bigger than a specified length. Wire-frame methods are context independent as the only inputs that impact the shape of the generated surface come from the altitude values at the vertices. This context independence is the opposite of what happens with fBm, which incorporate an infinite span of context dependence \cite{Mandelbrot1988}.


% TODO: Talk about creases
%\begin{quotation}
%"The "creases" sometimes encountered in stochastic subdivision can be related to the displacement noise variances and to the absence of autocorrelation information in Markovian subdivision."\cite{Lewis1987}
% \end{quotation}

\subsubsection{Tile Midpoint Displacement}

In tile midpoint displacement methods the surface is seen as a collection of tiles and the displacements are applied to points in the middle of every tile. This methods are context dependent.

\paragraph{Triangular tile midpoint displacement}
In \cite{Mandelbrot1988} Mandelbrot proposes a method of tile displacement with triangles. In contrast with the Carpenter's method (section \ref{sect:wireframeMD}), the displacement is applied to the midpoint of the triangles using: \[ H(P) = \frac{H(A) + H(B) + H(C)}{3} + rand \] where $A$, $B$ and $C$ are the triangle vertices, $P$ is the triangle midpoint  and $rand$ is the random displacement value.

\paragraph{Diamond-Square Algorithm}
The Diamond-Square algorithm \cite{Fournier1982} consists in the subdivision of quadrilaterals in two steps: in the first step, known as diamond, the midpoint of the square is displaced using a random value; in the second step, known as square, the midpoint of the original square sides are interpolated from the value of the two square vertices and the two closest diamond vertices and displaced by another random value. In practice this is a hybrid between wire-frame and tile displacement method as when the initial structure is composed of squares it is not enough to just displace the tiles midpoints but also interpolate the edges midpoints.


\paragraph{Square-square subdivision}
In \cite{Miller1986} Miller proposes a method adapted from the CAD/CAM field. This method, denominated Square-Square subdivision, generates new points that form a square which is half the size of the existing one using the proportions 9:3:3:1, in which the nearest points have the greater weight. 


\paragraph{Hexagonal tile midpoint displacement}
In \cite{Mandelbrot1988} Mandelbrot also proposes the use of hexagonal tiles in the displacement method. This was due to his belief that the nesting properties of the displacement methods, that is when the generated points are nested in the old structure, was the cause of the creasing effect. Given the hexagon's properties, a structure like this will never nest but, instead, will create a crumpled boundary that fails to catch the eye, contrary to the "creases" that stand out.

% NB: Solves creases

\subsection{Fourier Filtering}

Another method to generate fBm surfaces is using the Inverse Fourier Transform. This is done by generating a two dimensional Gaussian white noise signal, applying a $1/f^\beta$ low-pass filter in the frequency domain, and using the result of the Inverse Fourier Transform of the filtered signal as a height field \cite{Musgrave1989}. % This method has a time complexity $O(n \log{n})$ 

\subsection{Successive Random Additions}

The Successive Random Additions algorithm \cite{Saupe1988} builds on top of the midpoint displacement algorithm. If old points are reused in subsequent phases of subdivisions, they are displaced with a random variable with an appropriate distribution \cite{Musgrave1989}. In terms of complexity this method is comprised of approximately twice the number of additions of the midpoint displacement algorithm \cite{Voss1985}.

\subsection{Noise synthesis}

Noise synthesis consists in the addition of successive frequencies of tightly band-limited noises \cite{Musgrave1989}. This can be done using several noise algorithms, such as, Perlin Noise \cite{Perlin1985}, Simplex Nosie \cite{Perlin2002} or OpenSimplex Noise \cite{Spencer2015}.

\subsection{Generalized Stochastic Subdivision}

% TODO:		- Generalized Stochastic
All the methods previously presented have a basis function that is, usually, implicit in the generation algorithm. This basis function can affect the final surface in different ways: the saw-tooth wave in polygon subdivision methods explains the creasing effect and the sine wave in Fourier synthesis causes the terrains to become periodic. In \cite{Lewis1987} Lewis proposes a method that interpolates several local points based on a autocorrelation function \cite{Musgrave1993}, that is, this algorithm is able to generate surfaces with any basis function.

% \subsubsection{Multifractals}
% TODO: 	- Multifractals

%\begin{quotation}
%	"Multifractals may be heuristically defined as "fractals that require a multiplicity of measures, such as fractal dimension, to characterize them." Another heuristic definition is "heterogeneous fractals, the heterogeneity of which is invariant with scale. They are most easily thought of as fractals whose dimension varies with location; the key point is that they are heterogeneous -- not the same everywhere"\cite{Musgrave1994}
%\end{quotation}

%See \cite{Evertsz1992}.

% TODO: Erosion

% \section{Constrained Terrain Generation}

\section{Terrain Representation}

% TODO: TINs

\subsection{Height field}
The height field, in some literature denominated as height map, is one of the most used forms of representing terrains in Computer Graphics. A height field stores an altitude value at regular intervals using a two-dimensional array \cite{Ebert2003}, and, as such, can be saved as a grayscale image. Due to the fact that only one altitude value is retained for a pair of coordinates this method can only represent surfaces, that is, it does not support overhangs or caves. 

%\subsubsection{DEM file format}
% TODO: DEM
% TODO: Vector map
% TODO: DTM

%\begin{quotation}
%	"There are several common file formats for height field data. There is the DEM (digital elevation map) format of the U.S. Geological Survey (USGS) height fields, which contain measured elevation data corresponding to the "quad" topographic [...]" \cite[pg 494]{Ebert2003}
%\end{quotation}

% \section{Terrain Rendering}

% TODO: Rendering Optimizations (GPUGems article)
% TODO: Scene Management
% TODO: Lighting
% TODO: Shadows
\chapter{Methodology}

\section{Process Overview}

\section{Random Surface Generators}
% Why these and not others?
  \subsection{Fourier Filtering}
    \subsection{Noise Synthesis}
      \subsubsection{Perlin Noise}
      \subsubsection{Simplex Noise}

\section{Blending Process}

  \subsection{Modified Unsharp Filter}
    \subsubsection{Description} % Details and Usage
    \subsubsection{Parameters}
  \subsection{Result Normalization}
  \subsection{Surface Normals Calculation}
 
\include{software_architecture}
\chapter{Results} \label{chap:results}

  \section {Visual Presentation} % Before and After
  
    \begin{itemize}
    	\item Show some comparison between base surfaces and results with different parameters
    	\item Show some unreal engine 4 renderings of the results
    \end{itemize}
  
  \section {GPU vs CPU Computations Benchmarks}
  
    \begin{itemize}
    	\item Hardware Specification
    	\item Software Specification (including chrome version and flags)
    	\item Operations/second or time comparison
    	\item Ratios
    \end{itemize}
\chapter{Conclusions} \label{chap:conclusions}
  
  In this thesis a procedural terrain modelling method was proposed in the form of a hybrid process, in which, the 3d modeller creates a simplified base terrain comprised of the major features, and the developed system adds the high level details by generating a random surface, extracting it's details and adding them to the base terrain. 
  
  \section {Implemented Features}
  
    The system was implemented as a client-side single page web application using \textit{Angular.JS} and \textit{WebGL}. The surface enhancement process was developed, as well as, optimized using WebGL 2.0 as a platform to perform some computations and, as a result, some calculations are now performed between 10 and 20 times faster than with the CPU implementation. In order to preview the resultant terrain the \textit{three.js} library was used to render the terrain in an HTML 5 canvas. Furthermore the results can be exported as height maps or in a zip format supported by the implemented system which can be reimported and re-edited.
  
  \section {Further Work}
  
    Although the implemented system is capable of performing all the required operations some additional improvements could be researched, such as, other detail extraction methods, for example using different filters; usage of the base surface gradient in the blending process, more specifically, in the base surface mapping step in conjunction with the base surface height; additionally some more tools could be given to the user in order to provide more control over where to add details in the surface, as well as, to enable the usage of different method in distinct regions of the terrain.
    
    


% References - you can use BiBTeX inside the 'references' environment if you want.
% For short reference lists, you might find it just as easy to use the bibitem form.
\begin{references}
	\bibitem{Kruchten1995} 
Kruchten P. {Architectural Blueprints—The ‘4+1’ View Model of Software Architecture}. IEEE Software. 1995; 12(6): 42--50.

\bibitem{Mandelbrot1983} 
Mandelbrot BB. The Fractal Geometry of Nature. 1983.

\bibitem{Mandelbrot1968} 
Mandelbrot BB, Ness JV. Fractional Brownian motions, fractional noises and applications. SIAM review. 1968; Available at: http://epubs.siam.org/doi/abs/10.1137/1010093

\bibitem{Musgrave1993} 
Musgrave FK. Methods for Realistic Landscape Imaging. Thesis. Yale University; 1993. pp. 1--276. Available at: http://www.kenmusgrave.com/dissertation.pdf

\bibitem{Ebert2003} 
Ebert DS, Musgrave FK, Peachey D, Perlin K, Worley S, Mark WR, et al. Texturing and Modeling. Texturing and Modeling. Elsevier; 2003. 9 p. Available at: DOI:10.1016/B978-155860848-1/50050-4

\bibitem{Musgrave1989} 
Musgrave FK, Kolb CE, Mace RS. The synthesis and rendering of eroded fractal terrains. ACM SIGGRAPH Computer Graphics. New York, New York, USA: ACM Press; 1989; 23(3): 41--50. Available at: DOI:10.1145/74334.74337

\bibitem{Mandelbrot1988} 
Mandelbrot BB. Fractal landscapes without creases and with rivers. The science of fractal images. 1988. pp. 243--260. Available at: http://dl.acm.org/citation.cfm?id=61160 http://portal.acm.org/citation.cfm?id=61160

\bibitem{Fournier1982} 
Fournier A, Fussell D, Carpenter L. Computer rendering of stochastic models. Communications of the ACM. ACM; June 1982; 25(6): 371--384. Available at: DOI:10.1145/358523.358553

\bibitem{Lewis1987} 
Lewis JP. Generalized stochastic subdivision. ACM Transactions on Graphics. 1987; 6(3): 167--190. Available at: DOI:10.1145/35068.35069

\bibitem{Miller1986} 
Miller GSP. The definition and rendering of terrain maps. ACM SIGGRAPH Computer Graphics. 1986; 20(4): 39--48. Available at: DOI:10.1145/15886.15890

\bibitem{Saupe1988} 
Saupe D. Algorithms for random fractals. The Science of Fractal Images. New York, NY: Springer New York; 1988. pp. 71--136. Available at: DOI:10.1007/978-1-4612-3784-6\_2

\bibitem{Voss1985} 
Voss RF. Random Fractal Forgeries. Fundamental Algorithms for Computer Graphics. Berlin, Heidelberg: Springer Berlin Heidelberg; 1985. pp. 805--835. Available at: DOI:10.1007/978-3-642-84574-1\_34

\bibitem{Perlin1985} 
Perlin K. An image synthesizer. ACM SIGGRAPH Computer Graphics. 1985; 19(3): 287--296. Available at: DOI:10.1145/325165.325247

\bibitem{Perlin2002} 
Perlin K. Improving noise. ACM Transactions on Graphics. New York, New York, USA: ACM Press; 2002; 21(3): 2--3. Available at: DOI:10.1145/566654.566636

\bibitem{Spencer2015} 
Spencer K. OpenSimplexNoise.java. 2015. Available at: https://gist.github.com/KdotJPG/b1270127455a94ac5d19

\bibitem{Musgrave1994} 
Musgrave FK. 2 Procedural Fractal Terrains. Texturing and Modelling. A Procedural approach. 1994;  2.

\bibitem{Evertsz1992} 
Evertsz CJG, Mandelbrot BB. Multifractal measures. Chaos and Fractals. 1992. pp. 921--953. Available at: http://lipenreferences.googlecode.com/svn/trunk/Papers/Multifractals/1992Multifractal Measures.pdf

\bibitem{Google2016} 
Google. Material design. Google design guidelines. 2016. Available at: https://material.google.com

\bibitem{Lloyd2008} 
Lloyd DB, Boyd C, Govindaraju N. Fast Computation of General Fourier Transforms on GPUs. Microsoft Research; April 2008 p. 7. Available at: https://www.microsoft.com/en-us/research/publication/fast-computation-of-general-fourier-transforms-on-gpus/

\bibitem{Lyons2004} 
Lyons RG. Understanding digital signal processing. 2nd edn. Upper Saddle River, New Jersey: Prentice Hall; 2004.


\end{references}

%% Back matter
%
% This is where we include appendices and references

\chapter*{Appendices}
\addcontentsline{toc}{chapter}{Appendices}

\begin{appendices}	
	\section{User Manual}
  \begin{notes}
   	\item Explain each UI panel in detail
   	\item Insert activity diagram showing the main scenario
   	\item Possibly one diagram per scenario (See User Stories in Table \ref{table:user_stories})
  \end{notes}
	\newcolumntype{R}{>{\raggedleft\let\newline\\\arraybackslash\hspace{0pt}}m{5em}}

\begin{landscape}
	\pagestyle{lscape}
\section{Additional Results}
  
  In this appendix some extra results are presented. The results presented in Section \ref{app:sec:benchmarks} and Section \ref{app:sec:terrain_param} correspond to the values used to generate the plots and terrains from Sections \ref{sec:res:benchmarks} and \ref{sec:res:terrains} respectively.

  \subsection{Benchmarks} \label{app:sec:benchmarks}

\vspace*{\fill}

    \begin{table}[H]
	    \begin{adjustbox}{center}
%	    \footnotesize
    	\begin{tabular}{|l|R|R|R|R|R|R|R|}
    		\hline
    		& \multicolumn{1}{c|}{\textbf{16}} & \multicolumn{1}{c|}{\textbf{32}} & \multicolumn{1}{c|}{\textbf{64}} & \multicolumn{1}{c|}{\textbf{128}} & \multicolumn{1}{c|}{\textbf{256}} & \multicolumn{1}{c|}{\textbf{512}} & \multicolumn{1}{c|}{\textbf{1024}} \\ \hline
    		\textbf{FFT}                           & 4179.20                 & 3964.26                 & 3620.65                 & 2542.06                  & 1956.19                  & 560.92                   & 77.63                     \\ \hline
    		\textbf{EW Addition}       	 & 13821.15                & 13064.76                & 14438.87                & 9394.57                  & 7581.52                  & 3302.51                  & 1054.54                   \\ \hline
    		\textbf{EW Subtraction}    	 & 14867.98                & 13636.73                & 14472.53                & 9862.57                  & 8887.81                  & 3697.81                  & 1057.83                   \\ \hline
    		\textbf{EW Multiplication}   & 15354.87                & 14094.35                & 14613.53                & 10283.52                 & 8205.92                  & 3532.90                  & 1068.05                   \\ \hline
    		\textbf{Normalization}          & 626.63                  & 333.82                  & 338.54                  & 365.62                   & 349.89                   & 295.60                   & 292.79                    \\ \hline
    		\textbf{Perlin Noise}        & 456.57                  & 1115.75                 & 371.43                  & 385.58                   & 368.48                   & 256.84                   & 418.84                    \\ \hline
    		\textbf{Simplex Noise}       & 111.52                  & 85.36                   & 436.74                  & 84.26                    & 83.93                    & 42.67                    & 50.19                     \\ \hline
    		\textbf{Fourier Filtering}             & 101.38                  & 479.05                  & 432.66                  & 77.88                    & 91.32                    & 44.18                    & 51.32                     \\ \hline
    	\end{tabular}
    	\end{adjustbox}
    	\caption{GPU Benchmark Results in Operations per Second}
    \end{table}
    
      \vspace*{\fill}
      
    \newpage
    
      \vspace*{\fill}
    
    \begin{table}[H]
      \begin{adjustbox}{center}
%   		\footnotesize
    	\begin{tabular}{|l|R|R|R|R|R|R|R|}
    		\hline
    		& \multicolumn{1}{c|}{\textbf{16}} & \multicolumn{1}{c|}{\textbf{32}} & \multicolumn{1}{c|}{\textbf{64}} & \multicolumn{1}{c|}{\textbf{128}} & \multicolumn{1}{c|}{\textbf{256}} & \multicolumn{1}{c|}{\textbf{512}} & \multicolumn{1}{c|}{\textbf{1024}} \\ \hline
    		\textbf{FFT}                           & 43854.25                & 9920.98                 & 2266.42                 & 488.95                   & 55.52                    & 7.21                     & 1.34                      \\ \hline
    		\textbf{EW Addition}       	           & 278212.75               & 75512.41                & 21943.37                & 6107.86                  & 1589.63                  & 360.05                   & 82.28                     \\ \hline
    		\textbf{EW Subtraction}    	           & 164290.85               & 45383.68                & 11940.02                & 3248.37                  & 822.77                   & 105.84                   & 22.52                     \\ \hline
    		\textbf{EW Multiplication}             & 160549.50               & 44596.58                & 11622.45                & 3156.14                  & 806.14                   & 96.66                    & 23.84                     \\ \hline
    		\textbf{Normalization}                 & 84600.92                & 22334.53                & 5894.14                 & 1523.68                  & 382.28                   & 65.21                    & 11.72                     \\ \hline
    		\textbf{Perlin Noise}                  & 87500.00                & 24064.28                & 6114.19                 & 1586.12                  & 403.06                   & 96.43                    & 24.50                     \\ \hline
    		\textbf{Simplex Noise}                 & 5939.38                 & 1705.45                 & 476.55                  & 128.86                   & 33.67                    & 8.54                     & 2.22                      \\ \hline
    		\textbf{Fourier Filtering}             & 4923.12                 & 1429.14                 & 383.19                  & 102.44                   & 26.46                    & 6.69                     & 1.74                      \\ \hline
    	\end{tabular}
    	\end{adjustbox}
	    \caption{CPU Benchmark Results in Operations per Second}
    \end{table}
    
    \vspace*{\fill}
    
    \newpage

    \vspace*{\fill}
    
    \begin{table}[H]
      \begin{adjustbox}{center}
%      	\footnotesize
    	\begin{tabular}{|l|R|R|R|R|R|R|R|}
    		\hline
    		& \multicolumn{1}{c|}{\textbf{16}} & \multicolumn{1}{c|}{\textbf{32}} & \multicolumn{1}{c|}{\textbf{64}} & \multicolumn{1}{c|}{\textbf{128}} & \multicolumn{1}{c|}{\textbf{256}} & \multicolumn{1}{c|}{\textbf{512}} & \multicolumn{1}{c|}{\textbf{1024}} \\ \hline
    		\textbf{FFT}                         & 9.53\%                  & 39.96\%                 & 159.75\%                & 519.90\%                 & 3523.49\%                & 7777.27\%                & 5776.39\%                 \\ \hline
    		\textbf{EW Addition}       			 & 4.97\%                  & 17.30\%                 & 65.80\%                 & 153.81\%                 & 476.94\%                 & 917.24\%                 & 1281.69\%                 \\ \hline
    		\textbf{EW Subtraction}    			 & 9.05\%                  & 30.05\%                 & 121.21\%                & 303.62\%                 & 1080.23\%                & 3493.66\%                & 4697.81\%                 \\ \hline
    		\textbf{EW Multiplication}           & 9.56\%                  & 31.60\%                 & 125.74\%                & 325.83\%                 & 1017.93\%                & 3655.08\%                & 4479.68\%                 \\ \hline
    		\textbf{Normalization}        		 & 0.74\%                  & 1.49\%                  & 5.74\%                  & 24.00\%                  & 91.53\%                  & 453.29\%                 & 2498.49\%                 \\ \hline
    		\textbf{Perlin Noise}                & 0.52\%                  & 4.64\%                  & 6.07\%                  & 24.31\%                  & 91.42\%                  & 266.34\%                 & 1709.64\%                 \\ \hline
    		\textbf{Simplex Noise}               & 1.88\%                  & 5.01\%                  & 91.65\%                 & 65.39\%                  & 249.25\%                 & 499.42\%                 & 2261.13\%                 \\ \hline
    		\textbf{Fourier Filtering}           & 2.06\%                  & 33.52\%                 & 112.91\%                & 76.03\%                  & 345.20\%                 & 659.93\%                 & 2951.49\%                 \\ \hline
    	\end{tabular}
    \end{adjustbox}
    	\caption{GPU / CPU Benchmark Result Ratio}
    \end{table}
    
      \vspace*{\fill}

  \subsection{Terrain Parameters} \label{app:sec:terrain_param}
  
    \begin{table}[H]
    	\newcommand{\splineheight}{1.5em}
    	\newcommand\cincludegraphics[2][]{\raisebox{-0.3\height}{\includegraphics[#1]{#2}}}
    	\footnotesize
    	\centering
    	\begin{adjustbox}{center}
    	\input{param_table.tbl}
    \end{adjustbox}
    	\caption{Terrain parameters from Chapter \ref{chap:results}.}
    	\label{tab:res:terrain_param}
    \end{table}

\end{landscape}
\end{appendices}

\backmatter

\end{document}

